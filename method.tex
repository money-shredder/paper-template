\section{Method}\label{sec:method}

\subsection{Paper-specific Macros}\label{sub:method:macros}

It is recommended (and \textbf{compulsory!})
that you define macros
in \verb|common.tex|
for paper-specific macros,
using \verb|\newcommand|, \verb|\DeclareMathoOperator|, \etc{}
For example,
to define a macro \verb|\method{}| for the method name,
use \verb|\newcommand{\method}{Method}|.


\subsection{Cross References}\label{sub:method:crossref}

Use \verb|\label{sec:...}|
to label a section,
\verb|\label{sec:...:...}|
to label a subsection,
\etc{}

Use \verb|\label{fig:section:...}| to label a figure,
\verb|\label{fig:section:figure_name:...}| to label a subfigure,
and \verb|\label{tab:section:...}| to label a table.

\subsection{Figures}\label{sub:method:figures}

For figures,
use
\begin{verbatim}
\begin{figure}[ht]
    \centering
    \includegraphics[width=0.75\linewidth]{example}
    \caption{An example figure.}\label{fig:example}
\end{figure}
\end{verbatim}
to include a figure
from \verb|figures/example.pdf|.
It will be rendered as \Cref{fig:example}.
Use \verb|\subplot| to include subfigures.
For example,
to render \Cref{fig:subfigs},
use
\begin{verbatim}
\begin{figure}[ht]
    \centering
    % The optional [2] means 2 subfigures per row,
    % if not specified, it defaults to 1 subfigure per row.
    % The optional argument [trim=50pt 10pt 50pt 10pt, clip]
    % trims the figure by 50pt on the left, 10pt on the bottom,
    % 50pt on the right, and 10pt on the top,
    % and clips the figure to the specified size.
    % The argument {example} points to figures/example.pdf.
    % The argument {An example subfigure A.} is the caption.
    % The argument {fig:example:a} is the label.
    \subplot[2][
        trim=50pt 10pt 50pt 10pt, clip
    ]{example}{An example subfigure A.}{fig:example:a}
    \subplot[2][
        trim=50pt 10pt 50pt 10pt, clip
    ]{example}{An example subfigure B.}{fig:example:b}
    \caption{An example figure with subfigures.}\label{fig:example}
\end{figure}
\end{verbatim}

\begin{figure}[ht]
    \centering
    \includegraphics[width=0.75\textwidth]{example}
    \caption{An example figure.}\label{fig:example}
\end{figure}
\begin{figure}[ht]
    \centering
    % The optional [2] means 2 subfigures per row,
    % if not specified, it defaults to 1 subfigure per row.
    % The optional argument [trim=50pt 10pt 50pt 10pt, clip]
    % trims the figure by 50pt on the left, 10pt on the bottom,
    % 50pt on the right, and 10pt on the top,
    % and clips the figure to the specified size.
    % The argument {example} points to figures/example.pdf.
    % The argument {An example subfigure A.} is the caption.
    % The argument {fig:subfigs:a} is the label.
    \subplot[2][trim=50pt 10pt 50pt 10pt, clip]{example}{%
        An example subfigure A.
    }{fig:subfigs:a}%
    \subplot[2][trim=50pt 10pt 50pt 10pt, clip]{example}{%
        An example subfigure B.
    }{fig:subfigs:b}\hpad
    \caption{An example figure with subfigures.}\label{fig:subfigs}
\end{figure}


\subsection{Math Symbols and Formulae}\label{sub:method:mathematics}

It is recommended (and \textbf{compulsory!})
that you define macros
in \verb|common.tex|
for math symbols,
and these macros
can be customized easily later.
To standardize math symbols,
use the following to construct macros:
\begin{itemize}

    \item Random variables:
    \( \ra, \rb, \rc \).

    \item Random vectors:
    \( \rva, \rvb, \rvc \).

    \item Elements of random vectors:
    \( \erva, \ervb, \ervc \).

    \item Random matrices:
    \( \rmA, \rmB, \rmC \).

    \item Elements of random matrices:
    \( \ermA, \ermB, \ermC \).

    \item Vectors:
    \( \vzero, \vone, \va, \vb, \vc \).

    \item Elements of vectors:
    \( \eva, \evb, \evc \).

    \item Matrix:
    \( \mA, \mB, \mC \).

    \item Entries of a matrix:
    \( \emA, \emB, \emC \).

    \item Tensor:
    \( \tA, \tB, \tC \).

    \item Entries of a tensor:
    \( \etA, \etB, \etC \).

    \item Sets:
    \( \sA, \sB, \sC \).

    \item Underlying data generating distribution
    \( \pdata \),
    empirical distribution defined by the training set
    \( \ptrain \),
    model distribution \( \pmodel, \ptildemodel \),
    autoencoder distributions \( \pencode, \pdecode, \precons \).

    \item Functions:
    Expectation \( \E \),
    loss \( \Ls \),
    KL divergence \( \KL \),
    Variance \( \Var \),
    \etc{}

    \item Norms:
    \( \normlzero \), \( \normlone \), \( \normltwo \),
    \( \normlp \), \( \normmax \).

    \item Others:
    \( \argmax \), \( \argmin \), \( \sign \)

\end{itemize}

Use \verb|\( ... \)| to surround inline math,
\verb|\begin{equation}| and \verb|\end{equation}|
for display-style equations,
and
\begin{verbatim}
    \begin{equation}
        \begin{aligned}
            ...
        \end{aligned}\label{eq:abc}
    \end{equation}
\end{verbatim}
for multi-line equations.
Use \verb|\cref{eq:abc}| to reference the equation above.
